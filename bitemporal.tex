\documentclass{article}

\title{Document Oriented Management of Bitemporal Data}

\author{me}

\begin{document}

\maketitle

\begin{abstract}
  Bitemporal data mangagement is required in industries ranging from insurances
  over banking to the public sector. As a result, the SQL 2011 standard has
  introduced bitemporal data management facilities for relational databases,
  which address many, but not all bitemporal management requirements. This
  paper shows that document oriented databases are beneficial when it comes to
  managing bitemporal data, since document oriented data management reduces the
  number of joins required in queries.
 
  As a further contribution, this paper introduces an object oriented interface
  for bitemporal databases that fulfills all requirements for bitemporal data
  management. Finally, a small open-source prototype is introduced that shows
  that this interface is easy to implement on top of a document oriented data
  base in practice.
\end{abstract}

\section{Related Work}



\section{Bitemporal Data Management in Practice}

\section{Bitemporal Documents and Objects}

\section{Bitemporal Document Databases}

\end{document}
